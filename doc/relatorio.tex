\documentclass{llncs}
\usepackage{graphicx}
\usepackage[portuguese]{babel}
\usepackage[utf8]{inputenc}
 
\begin{document}
\title{Aplicação de programação orientada à organização de agentes no Multi-Agent Programming Contest}
\author{Mariana Ramos Franco e Rafael Barbolo Lopes}
\institute{Escola Politécnica da Universidade de São Paulo \\ Departamento de Engenharia de Computação e Sistemas Digitais \\ PCS 5703 - Sistemas Multi-Agentes}
\maketitle

\begin{abstract}
blablablabla...
\end{abstract}


\section{Introdução}

% Breve descrição do problema, bem como de software (modelos, plataformas, etc) e hardware utilizados.


\section{Análise e especificação do SMA}

% Descrição do método adotado para o desenvolvimento do SMA; especificação dos requisitos do SMA e especificação dos componentes do SMA (agentes, organização, interações, etc) segundo o método de desenvolvimento adotado.

% \begin{enumerate}
%  \item How is your system  specified and designed?
%  \item Did you use any existing multi-agent system methodology such as Prometheus, Gaia or Tropos?
%  \item Which strategies and algorithms do you plan to use?
%  \item How are the following agent features implemented: 
% 			\emph{autonomy}, \emph{proactiveness} and \emph{communication} 
%			\emph{team working}, and \emph{coordination}?
%  \item Is your system a truly \textbf{multi}-agent system or rather a centralised system in disguise?
% \end{enumerate}

Para o desenvolvimento do SMA foi utilizado um método situacional construido sob medida para o projeto, a partir de fragmentos de métodos capturados de três abordagens de desenvolvimento: USDP \cite{USDP}, Gaia \cite{GAIA} e MOISE+ \cite{MOISE}.


\section{Arquitetura e design do SMA}

% Design dos componentes do SMA e descrição da arquitetura do SMA, segundo o método de desenvolvimento adotado.

% \begin{enumerate}
%  \item Which programming language do you plan to use to implement the multi-agent system?
%  \item How would you map the designed architecture (both multi-agent and individual agent architectures)
%    to programming codes, i.e., how would you implement specific agent-oriented
%    concepts and designed artifacts using the programming language?
%  \item Which development platform, tools and techniques are you planning to use?
% \end{enumerate}
% Please give reasons why you have chosen the methods explained above.

\section{Linguagens de programação e plataforma de execução}

% Tecnologia SMA utilizada, características da implementação, etc.

\section{Estratégia para time de agentes}

% Descrição do algoritmo de deslocamento dos agentes, das estratégias de coordenação e de otimização de tarefas, etc.

% \begin{enumerate}
%    \item Describe the navigation algorithms:
%        \begin{itemize}
%            \item obstacle avoiding
%            \item strategy for finding and herding cows
%            \item opponent blocking
%        \end{itemize}
%    \item Describe the team coordination strategy (if any)
%    \item Does your team strategy use some distributed optimization
%        technique w.r.t. e.g.  minimizing distances walked by the
%        agents?
%    \item Describe and discuss the information exchanged (and shared) in
%        the agent team.
%    \item Describe the communication strategy in the agent team. Can you
%        estimate the communication complexity in your approach?
%    \item Did your system do some background processing? Under background
%        processing we understand some computation which happened while agents of
%        the team were \textit{idle}, i.e. between sending an action
%        message to the simulation server and receiving a perception
%        message for the subsequent simulation step.
%    \item Possibly discuss additional technical details of your system like
%        e.g. failure/crash recovery and alike.
% \end{enumerate}


\section{Características técnicas}

% Descrição de características técnicas do SMA desenvolvido, tais como estabilidade e recuperação de falhas.

\section{Discussão e conclusão}

% Comentários sobre a facilidade/dificuldade de utilização do método de desenvolvimento adotado, comentários sobre a facilidade e ou dificuldade do modelo organizacional para este domínio, bem como possíveis extensões que poderiam ser realizadas na solução proposta.

% In this section please expand a bit on your experience with the
% contest organization, the proposed scenario, and the setup of the
% actual contest. Please indicate what do you see as pros/cons of
% participating in the Contest with respect to your research in the
% field.
% \begin{enumerate}
%    \item Critical discussion of your approach to the development of the
%        agent team.
%    \item Will you gain some insights/experiences into developing multi-agent system in
%        the course of participating in the Agent Contest so far? If so,
%        what kind of?
%    \item Do you think that the Agent Contest can serve its
%        purpose to provide a testbed for state-of-the-art multi-agent
%        systems development frameworks?
%    \item Describe and discuss possible problems that you face in choosing approach, programming
%        platform or technical infrastructure to participate in this contest?
%\end{enumerate}

\bibliographystyle{plain}
\bibliography{myrefs}

\end{document}

